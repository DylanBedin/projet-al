\chapter{Sérialisation: sauvegarde et chargement des fichiers}


Deux des fonctionnalités demandées lors de ce projet furent la capacité à
sauvegarder les formes dans un fichier, puis, pour plus tard, les charger
depuis ce fichier.\newline
La sérialisation des formes contenues dans la whiteboard se déclenche lorsque
le bouton ``Save as'' est enclenché, en
commencant d'abord par créer/choisir un fichier grâce à la classe JFileChooser,
pour ensuite l'ouvrir et écrire dessus la liste des formes contenue dans la
whiteboard. Nous obtenons donc un fichier de format binaire contenant chacune des
formes présentes, donc leurs positions y compris, lors de l'appel à la
sérialisation. Le controller gère à lui tout seul ces actions, le modèle est
appelé seulement pour récupérer la liste des formes.\newline
Le chargement d'une liste de formes consiste donc à faire la procédure
précédente, mais à l'envers. Lorsque le bouton ``Load'' est enclenché, le
controller va encore appelé un JFileChooser qui va nous permettre de choisir
le fichier binaire désiré. On va ensuite ouvrir ce fichier à la lecture et
caster son contenu en une ``ArrayList<Ishape>''. Il suffit maintenant d'écraser la
liste des formes actuelles contenues dans le model via un clear(), pour
ensuite ajouter une par une les formes contenues dans la liste extraite du
fichier. Une fois cette opération terminée, nous prévenons le model du
changement apporté à la liste des shapes, qui va immédiatement remplacer les
shapesFX présentes par les shapesFX de notre nouvelle liste.

