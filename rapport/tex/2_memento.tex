\chapter{Memento: undo/redo}

Le choix a été fait de mettre en place un Memento afin de pouvoir annuler et
refaire des opérations. Pour cela, nous allons stocker des Model
dans des piles d'objets Memento. Nous avons ainsi créé 2 classes. La classe
Memento sert à stocker un Model et à pouvoir y accéder. La classe
MementoOriginator permet de stocker un Model et s'occupera de sauvegarder
ce dernier dans l'objet Memento. Enfin, la clase Model stocke 2 piles, une pile
servant à effectuer des undo, l'autre servant à faire les redo. A l'instanciation, les deux sont vides. A chaque fois que l'on notifie les
observeurs, nous faisons appel non pas directement à la méthode ``notifyObservers'' de la classe Observable mais à une méthode ayant un booléen
``memento''. Si celui-ci est set à ``true'', nous faisons appel à
la méthode ``saveMemento'' qui sauvegarde le Model courant et l'empile dans la pile ``undoStack''. Ainsi, dès que nous voulons effectuer un undo,
il suffit de dépiler cette pile. Or, afin de permettre de faire l'opération redo, il suffit, au moment du dépilement d'''undoStack'', d'empiler dans
la pile redo. On dépilera cette dernière afin de faire l'opération redo. Cependant, à chaque fois que l'on empile dans la pile
``undo'',  l'idée est de vider la pile redo afin de ne pas obtenir des incohérences si l'on mêle des undo et des redo.
